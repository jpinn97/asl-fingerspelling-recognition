\section{Introduction}
\subsection{Background}
\begin{itemize}
    \item why is this topic (ASL) important? (e.g., accessibility, communication)
    Sign language is the primary form of communication for the deaf and hard of hearing community. It allows communication when the spoken language is not possible, and or when the speaker or receiver is deaf or hard of hearing.
    Depending on the situation, and like any language, it requires both parties to be fluent in the language to communicate effectively. However, this is not always the case. American Sign Language(ASL) is a complete, complex language that employs signs made with the hands and other movements, including facial expressions and postures of the body, and is used natively in the
    United States of America and globally by many individuals. Whilst no attempt has officially been made to survey the language, and most current estimates are based off of historical surveys that prove to be inaccurate\cite{mitchellHowManyPeople2006}. It is estimated that there are over 1 million signers\cite{AmericanSignLanguage}, but others estimates are as high as 2 million\cite{mitchellHowManyPeople2006}.
    ASL coommunicates through a variety of means including gestures, non-manual markers and lexical signs. The most understood are lexical vocavulary, each corresponding to a word or morphene. Gestures and non-manual markers such as facial expression can complement and convery more interactive or meaningful lexical signs. Additional constructs include usage of space, role shifting and classifiers.
    \item what is the specific problem being addressed? (e.g., fingerspelling recognition) (bridges the gap in communication and enhances the learning/usage of ASL.) (makes ai more accessible to this audience?)
    ASL , 
    \item what is the impact of an AI recogniser for this problem? (e.g., enables real-time communication, improves accessibility, etc.) (can this effect broader technology?)
\end{itemize}

\subsection{Purpose}

\begin{itemize}
    \item what is the purpose of the review?
    \item (e.g., to identify the state of the art in ASL fingerspelling recognition)
    \item (to identify the challenges and opportunities in ASL fingerspelling recognition)
    \item (to identify the most promising techniques for ASL fingerspelling recognition)
    \item * primary purpose is to build our own model, but we need to know what's out there first. *

\end{itemize}
\subsection{Scope}
\begin{itemize}
    \item what is the scope of the review? (e.g., ASL fingerspelling recognition) (what is the scope of the problem? (e.g., real-time recognition of fingerspelling gestures) (what is the scope of the solution? (e.g., image-based recognition of fingerspelling gestures) (what is the scope of the evaluation? (e.g., accuracy, speed, etc.)
    \item what is the scope of the literature? (e.g., papers published in the last 5 years) (what is the scope of the sources? (e.g., peer-reviewed journal articles, conference papers, etc.)
    \item what we're not covering.
    \item only recognition and translation of ASL *fingerspelling* (not full ASL).
    \item specific the application/methodology (e.g., video-based recognition of fingerspelling gestures) (live/stream???)
\end{itemize}

\subsection{Research Questions}
\begin{itemize}
    \item RQ1:
    \item RQ2:
    \item RQ3:
    \item RQ4:
    \item RQ5:
    \item RQ6:
\end{itemize}

\subsection{Methodology}
\begin{itemize}
    \item how was the literature identified? (e.g., search terms, databases, etc.)
    \item how was the literature evaluated? (e.g., inclusion/exclusion criteria, etc.)
    \item how was the literature synthesised? (e.g., thematic analysis, etc.)
\end{itemize}

\subsection{Report Structure}
- what is the structure of the report? (e.g., introduction, literature review, etc.) (aka roadmap?)

\subsection{Conclude Introduction}

\begin{itemize}
    \item By understanding, the insight gained from this review will be used to inform the design and implementation of our own model.
    \item * evidence based approach is neccessary to be impactful *

\end{itemize}
\section{Historical Context}
\begin{itemize}
    \item Evolution of ASL fingerspelling recognition
    \item Major milestones and breakthroughs
\end{itemize}

\section{Methods and Techniques}
\begin{itemize}
    \item Overview of methods used in the literature
    \item Image/Video-based Approaches
    \item Framework-based Approaches (e.g., MediaPipe)
    \item Hybrid methods
    \item Evaluation metrics commonly used
\end{itemize}

\section{Challenges in ASL Fingerspelling Recognition}
\begin{itemize}
    \item Technical challenges (e.g., diverse handshapes, background noise)
    \item Data challenges (e.g., lack of large labeled datasets)
    \item Real-world challenges (e.g., different lighting conditions, user variability)
\end{itemize}

\section{Overcoming Obstacles}
\begin{itemize}
    \item Techniques to enhance accuracy
    \item Data augmentation strategies
    \item Transfer learning and pre-trained models
\end{itemize}

\section{State of the Art/Case Studies}
\begin{itemize}
    \item Most recent and influential works in the field
    \item Comparison of different methods' performance
    \item Real-world applications and success stories
\end{itemize}

\section{Future Directions and Open Challenges}
\begin{itemize}
    \item Emerging trends in the field
    \item Areas that need further research
    \item Potential impact of advancements (e.g., in deep learning)
\end{itemize}

\section{Ethical and Societal Considerations}
\begin{itemize}
    \item Data privacy concerns
    \item Bias and fairness in ASL recognition models
    \item Implications for the deaf and hard of hearing community

\end{itemize}

\section{Main Conclusion}
\begin{itemize}
    \item Summary of the main findings of the review
    \item Reiteration of the importance of the topic
\end{itemize}