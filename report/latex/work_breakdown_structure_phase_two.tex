\begin{tikzpicture}[
                level 1/.style={sibling distance=30mm, level distance=25mm},
                level 2/.style={sibling distance=15mm},
                level 3/.style={sibling distance=15mm},
                edge from parent/.style={->,draw},
                >=latex]
        \tikzset{
                basic/.style  = {draw, text width=4cm, drop shadow, font=\sffamily\scriptsize, rectangle},
                root/.style   = {basic, rounded corners=2pt, thin, align=center,
                                fill=green!30},
                level 2/.style = {basic, rounded corners=6pt, thin,align=center, fill=green!60,
                                text width=5em},
                level 3/.style = {basic, thin, align=center, fill=pink!60, text width=6em}
        }
        % root of the the initial tree, level 1
        \node[root] {ASL Fingerspelling Recognition Project: Build Application}
        % The first level, as children of the initial tree
        child {node[level 2] (c1) {Requirements Gathering}}
        child {node[level 2] (c2) {Scope Definition}}
        child {node[level 2] (c3) {Technical Feasibility Study}}
        child {node[level 2] (c4) {User Experience Design}}
        child {node[level 2] (c5) {Development Environment Setup}}
        child {node[level 2] (c6) {Prototype Development}}
        child {node[level 2] (c7) {Testing}}
        child {node[level 2] (c8) {Iterative Feedback and Improvement}}
        child {node[level 2] (c9) {Documentation}};

        % The second level, relatively positioned nodes
        \begin{scope}[every node/.style={level 3}]
                % Tasks for Requirements Gathering
                \node [below of = c1, xshift=15pt] (c11) {Conduct Qualitive Research};
                \node [below of = c11] (c12) {Host Focus Groups};
                \node [below of = c12] (c13) {Research Existing Kiosks};

                % Tasks for Scope Definition
                \node [below of = c2, xshift=15pt] (c21) {Write Project Charter};
                \node [below of = c21] (c22) {Define MVP Features};
                \node [below of = c22] (c23) {Create Project Plan};

                % Tasks for Technical Feasibility Study
                \node [below of = c3, xshift=15pt] (c31) {Assess Hardware};
                \node [below of = c31] (c32) {Analyze Integration Points};

                % Tasks for User Experience Design
                \node [below of = c4, xshift=15pt] (c41) {Sketch Wireframes};
                \node [below of = c41] (c42) {Conduct Usability Testing};

                % Tasks for Development Environment Setup
                \node [below of = c5, xshift=15pt] (c51) {Select Languages/Frameworks};
                \node [below of = c51] (c52) {Configure Environments};
                \node [below of = c52] (c53) {Establish Tools};

                % Tasks for Prototype Development
                \node [below of = c6, xshift=15pt] (c61) {Implement Core Functions};
                \node [below of = c61] (c62) {Develop Front-end};
                \node [below of = c62] (c63) {Integrate ASL Module};

                % Tasks for Testing
                \node [below of = c7, xshift=15pt] (c71) {Develop Test Cases};
                \node [below of = c71] (c72) {Perform Testing};
                \node [below of = c72] (c73) {Fix Bugs};
                % Tasks for Iterative Feedback and Improvement
                \node [below of = c8, xshift=15pt] (c81) {Gather User Feedback};
                \node [below of = c81] (c82) {Refine Features};
                \node [below of = c82] (c83) {Update Project Plan};

                % Tasks for Documentation
                \node [below of = c9, xshift=15pt] (c91) {Document Code};
                \node [below of = c91] (c92) {Write Manual};
        \end{scope}

        % lines from each level 1 node to every one of its "children"
        % Loop over each parent
        % Loop over each parent
        \foreach \parent in {1,...,9} {
                        % Draw lines to child nodes that have been defined
                        \draw[->] (c\parent.195) |- (c\parent1.west);
                        \draw[->] (c\parent.195) |- (c\parent2.west);
                        % Check for exceptions with different number of children
                        \ifnum\parent=9
                                % Parent 9 only has 2 children
                        \else
                                \ifnum\parent=4
                                        % Parent 4 has 4 children
                                        \draw[->] (c\parent.195) |- (c\parent2.west);
                                \else
                                        \ifnum\parent=3
                                                % Parent 3 has 4 children
                                                \draw[->] (c\parent.195) |- (c\parent2.west);
                                        \else
                                                % All other parents have 3 children
                                                \draw[->] (c\parent.195) |- (c\parent3.west);
                                        \fi
                                \fi
                        \fi
                }
\end{tikzpicture}
